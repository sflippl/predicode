%% Generated by Sphinx.
\def\sphinxdocclass{report}
\documentclass[letterpaper,10pt,english]{sphinxmanual}
\ifdefined\pdfpxdimen
   \let\sphinxpxdimen\pdfpxdimen\else\newdimen\sphinxpxdimen
\fi \sphinxpxdimen=.75bp\relax

\PassOptionsToPackage{warn}{textcomp}
\usepackage[utf8]{inputenc}
\ifdefined\DeclareUnicodeCharacter
% support both utf8 and utf8x syntaxes
  \ifdefined\DeclareUnicodeCharacterAsOptional
    \def\sphinxDUC#1{\DeclareUnicodeCharacter{"#1}}
  \else
    \let\sphinxDUC\DeclareUnicodeCharacter
  \fi
  \sphinxDUC{00A0}{\nobreakspace}
  \sphinxDUC{2500}{\sphinxunichar{2500}}
  \sphinxDUC{2502}{\sphinxunichar{2502}}
  \sphinxDUC{2514}{\sphinxunichar{2514}}
  \sphinxDUC{251C}{\sphinxunichar{251C}}
  \sphinxDUC{2572}{\textbackslash}
\fi
\usepackage{cmap}
\usepackage[T1]{fontenc}
\usepackage{amsmath,amssymb,amstext}
\usepackage{babel}



\usepackage{times}
\expandafter\ifx\csname T@LGR\endcsname\relax
\else
% LGR was declared as font encoding
  \substitutefont{LGR}{\rmdefault}{cmr}
  \substitutefont{LGR}{\sfdefault}{cmss}
  \substitutefont{LGR}{\ttdefault}{cmtt}
\fi
\expandafter\ifx\csname T@X2\endcsname\relax
  \expandafter\ifx\csname T@T2A\endcsname\relax
  \else
  % T2A was declared as font encoding
    \substitutefont{T2A}{\rmdefault}{cmr}
    \substitutefont{T2A}{\sfdefault}{cmss}
    \substitutefont{T2A}{\ttdefault}{cmtt}
  \fi
\else
% X2 was declared as font encoding
  \substitutefont{X2}{\rmdefault}{cmr}
  \substitutefont{X2}{\sfdefault}{cmss}
  \substitutefont{X2}{\ttdefault}{cmtt}
\fi


\usepackage[Bjarne]{fncychap}
\usepackage{sphinx}

\fvset{fontsize=\small}
\usepackage{geometry}

% Include hyperref last.
\usepackage{hyperref}
% Fix anchor placement for figures with captions.
\usepackage{hypcap}% it must be loaded after hyperref.
% Set up styles of URL: it should be placed after hyperref.
\urlstyle{same}
\addto\captionsenglish{\renewcommand{\contentsname}{Contents:}}

\usepackage{sphinxmessages}
\setcounter{tocdepth}{1}


% Jupyter Notebook code cell colors
\definecolor{nbsphinxin}{HTML}{307FC1}
\definecolor{nbsphinxout}{HTML}{BF5B3D}
\definecolor{nbsphinx-code-bg}{HTML}{F5F5F5}
\definecolor{nbsphinx-code-border}{HTML}{E0E0E0}
\definecolor{nbsphinx-stderr}{HTML}{FFDDDD}
% ANSI colors for output streams and traceback highlighting
\definecolor{ansi-black}{HTML}{3E424D}
\definecolor{ansi-black-intense}{HTML}{282C36}
\definecolor{ansi-red}{HTML}{E75C58}
\definecolor{ansi-red-intense}{HTML}{B22B31}
\definecolor{ansi-green}{HTML}{00A250}
\definecolor{ansi-green-intense}{HTML}{007427}
\definecolor{ansi-yellow}{HTML}{DDB62B}
\definecolor{ansi-yellow-intense}{HTML}{B27D12}
\definecolor{ansi-blue}{HTML}{208FFB}
\definecolor{ansi-blue-intense}{HTML}{0065CA}
\definecolor{ansi-magenta}{HTML}{D160C4}
\definecolor{ansi-magenta-intense}{HTML}{A03196}
\definecolor{ansi-cyan}{HTML}{60C6C8}
\definecolor{ansi-cyan-intense}{HTML}{258F8F}
\definecolor{ansi-white}{HTML}{C5C1B4}
\definecolor{ansi-white-intense}{HTML}{A1A6B2}
\definecolor{ansi-default-inverse-fg}{HTML}{FFFFFF}
\definecolor{ansi-default-inverse-bg}{HTML}{000000}

% Define an environment for non-plain-text code cell outputs (e.g. images)
\makeatletter
\newenvironment{nbsphinxfancyoutput}{%
    % Avoid fatal error with framed.sty if graphics too long to fit on one page
    \let\sphinxincludegraphics\nbsphinxincludegraphics
    \nbsphinx@image@maxheight\textheight
    \advance\nbsphinx@image@maxheight -2\fboxsep   % default \fboxsep 3pt
    \advance\nbsphinx@image@maxheight -2\fboxrule  % default \fboxrule 0.4pt
    \advance\nbsphinx@image@maxheight -\baselineskip
\def\nbsphinxfcolorbox{\spx@fcolorbox{nbsphinx-code-border}{white}}%
\def\FrameCommand{\nbsphinxfcolorbox\nbsphinxfancyaddprompt\@empty}%
\def\FirstFrameCommand{\nbsphinxfcolorbox\nbsphinxfancyaddprompt\sphinxVerbatim@Continues}%
\def\MidFrameCommand{\nbsphinxfcolorbox\sphinxVerbatim@Continued\sphinxVerbatim@Continues}%
\def\LastFrameCommand{\nbsphinxfcolorbox\sphinxVerbatim@Continued\@empty}%
\MakeFramed{\advance\hsize-\width\@totalleftmargin\z@\linewidth\hsize\@setminipage}%
}{\par\unskip\@minipagefalse\endMakeFramed}
\makeatother
\newbox\nbsphinxpromptbox
\def\nbsphinxfancyaddprompt{\ifvoid\nbsphinxpromptbox\else
    \kern\fboxrule\kern\fboxsep
    \copy\nbsphinxpromptbox
    \kern-\ht\nbsphinxpromptbox\kern-\dp\nbsphinxpromptbox
    \kern-\fboxsep\kern-\fboxrule\nointerlineskip
    \fi}
\newlength\nbsphinxcodecellspacing
\setlength{\nbsphinxcodecellspacing}{0pt}

% Define support macros for attaching opening and closing lines to notebooks
\newsavebox\nbsphinxbox
\makeatletter
\newcommand{\nbsphinxstartnotebook}[1]{%
    \par
    % measure needed space
    \setbox\nbsphinxbox\vtop{{#1\par}}
    % reserve some space at bottom of page, else start new page
    \needspace{\dimexpr2.5\baselineskip+\ht\nbsphinxbox+\dp\nbsphinxbox}
    % mimick vertical spacing from \section command
      \addpenalty\@secpenalty
      \@tempskipa 3.5ex \@plus 1ex \@minus .2ex\relax
      \addvspace\@tempskipa
      {\Large\@tempskipa\baselineskip
             \advance\@tempskipa-\prevdepth
             \advance\@tempskipa-\ht\nbsphinxbox
             \ifdim\@tempskipa>\z@
               \vskip \@tempskipa
             \fi}
    \unvbox\nbsphinxbox
    % if notebook starts with a \section, prevent it from adding extra space
    \@nobreaktrue\everypar{\@nobreakfalse\everypar{}}%
    % compensate the parskip which will get inserted by next paragraph
    \nobreak\vskip-\parskip
    % do not break here
    \nobreak
}% end of \nbsphinxstartnotebook

\newcommand{\nbsphinxstopnotebook}[1]{%
    \par
    % measure needed space
    \setbox\nbsphinxbox\vbox{{#1\par}}
    \nobreak % it updates page totals
    \dimen@\pagegoal
    \advance\dimen@-\pagetotal \advance\dimen@-\pagedepth
    \advance\dimen@-\ht\nbsphinxbox \advance\dimen@-\dp\nbsphinxbox
    \ifdim\dimen@<\z@
      % little space left
      \unvbox\nbsphinxbox
      \kern-.8\baselineskip
      \nobreak\vskip\z@\@plus1fil
      \penalty100
      \vskip\z@\@plus-1fil
      \kern.8\baselineskip
    \else
      \unvbox\nbsphinxbox
    \fi
}% end of \nbsphinxstopnotebook

% Ensure height of an included graphics fits in nbsphinxfancyoutput frame
\newdimen\nbsphinx@image@maxheight % set in nbsphinxfancyoutput environment
\newcommand*{\nbsphinxincludegraphics}[2][]{%
    \gdef\spx@includegraphics@options{#1}%
    \setbox\spx@image@box\hbox{\includegraphics[#1,draft]{#2}}%
    \in@false
    \ifdim \wd\spx@image@box>\linewidth
      \g@addto@macro\spx@includegraphics@options{,width=\linewidth}%
      \in@true
    \fi
    % no rotation, no need to worry about depth
    \ifdim \ht\spx@image@box>\nbsphinx@image@maxheight
      \g@addto@macro\spx@includegraphics@options{,height=\nbsphinx@image@maxheight}%
      \in@true
    \fi
    \ifin@
      \g@addto@macro\spx@includegraphics@options{,keepaspectratio}%
    \fi
    \setbox\spx@image@box\box\voidb@x % clear memory
    \expandafter\includegraphics\expandafter[\spx@includegraphics@options]{#2}%
}% end of "\MakeFrame"-safe variant of \sphinxincludegraphics
\makeatother



\title{predicode}
\date{Aug 27, 2019}
\release{0.0.0.9000}
\author{Samuel Lippl}
\newcommand{\sphinxlogo}{\vbox{}}
\renewcommand{\releasename}{Release}
\makeindex
\begin{document}

\pagestyle{empty}
\sphinxmaketitle
\pagestyle{plain}
\sphinxtableofcontents
\pagestyle{normal}
\phantomsection\label{\detokenize{index::doc}}


\sphinxhref{https://travis-ci.org/sflippl/predicode}{\sphinxincludegraphics{{/home/sflippl/Documents/predicode/predicode/docs/doctrees/images/7256f7aae32b509954da3c73c160d223b834acbc/predicode}.svg}}

\sphinxhref{https://coveralls.io/github/sflippl/predicode?branch=master}{\sphinxincludegraphics{{/home/sflippl/Documents/predicode/predicode/docs/doctrees/images/951e4f9d39f518fb62b652a6589275328715ff0c/badge}.svg}}

\noindent\sphinxincludegraphics{{/home/sflippl/Documents/predicode/predicode/docs/doctrees/images/084f3f18e49bbd498e999d6e8a4f8c85433a232c/031167cf2618c18d14cd822be082033472892626}.7--dev-blue}

The package ‘predicode’ intends to provide a consistent Tensorflow interface to fitting hierarchical predictive coding models, syntactically guided by keras models.

Get started by 60 second introductions to predicode’s base functionality!


\chapter{Get Started}
\label{\detokenize{usage/get_started:Get-Started}}\label{\detokenize{usage/get_started::doc}}
\sphinxhref{https://colab.research.google.com/github/sflippl/predicode/blob/master/usage/get\_started.ipynb}{\sphinxincludegraphics{{/home/sflippl/Documents/predicode/predicode/docs/doctrees/images/1cbef8f858ac0e12d0cf2cb67076bcaff7e5a3c8/blue}.svg}}

{
\sphinxsetup{VerbatimColor={named}{nbsphinx-code-bg}}
\sphinxsetup{VerbatimBorderColor={named}{nbsphinx-code-border}}
\begin{sphinxVerbatim}[commandchars=\\\{\}]
\llap{\color{nbsphinxin}[1]:\,\hspace{\fboxrule}\hspace{\fboxsep}}\PYG{k}{try}\PYG{p}{:}
    \PYG{k+kn}{import} \PYG{n+nn}{predicode} \PYG{k}{as} \PYG{n+nn}{pc}
\PYG{k}{except}\PYG{p}{:}
    \PYG{o}{!}pip install git+https://github.com/sflippl/predicode
    \PYG{k+kn}{import} \PYG{n+nn}{predicode} \PYG{k}{as} \PYG{n+nn}{pc}
\PYG{k}{try}\PYG{p}{:}
    \PYG{k+kn}{import} \PYG{n+nn}{lazytools\PYGZus{}sflippl} \PYG{k}{as} \PYG{n+nn}{lazytools}
\PYG{k}{except}\PYG{p}{:}
    \PYG{o}{!}pip install git+https://github.com/sflippl/lazytools
    \PYG{k+kn}{import} \PYG{n+nn}{lazytools\PYGZus{}sflippl} \PYG{k}{as} \PYG{n+nn}{lazytools}
\end{sphinxVerbatim}
}


\section{A minimal model in 60 seconds}
\label{\detokenize{usage/get_started:A-minimal-model-in-60-seconds}}
A minimal hierarchical predictive coding model consists of a single, densely and linearly connected hidden layer. In itself, it amounts to a simple PCA, but already has interesting consequences, both in its outcome, which is able to explain neural effects such as endstopping, and in its learning trajectory.

To get some data that can well be approximated by a PCA, call pc.DecayingMultiNormal:

{
\sphinxsetup{VerbatimColor={named}{nbsphinx-code-bg}}
\sphinxsetup{VerbatimBorderColor={named}{nbsphinx-code-border}}
\begin{sphinxVerbatim}[commandchars=\\\{\}]
\llap{\color{nbsphinxin}[2]:\,\hspace{\fboxrule}\hspace{\fboxsep}}\PYG{n}{art} \PYG{o}{=} \PYG{n}{pc}\PYG{o}{.}\PYG{n}{decaying\PYGZus{}multi\PYGZus{}normal}\PYG{p}{(}\PYG{n}{dimensions}\PYG{o}{=}\PYG{l+m+mi}{10}\PYG{p}{,} \PYG{n}{size}\PYG{o}{=}\PYG{l+m+mi}{100}\PYG{p}{)}
\PYG{n}{lazytools}\PYG{o}{.}\PYG{n}{matrix\PYGZus{}heatmap}\PYG{p}{(}\PYG{n}{art}\PYG{p}{,} \PYG{n}{pole}\PYG{o}{=}\PYG{l+m+mi}{0}\PYG{p}{)}
\end{sphinxVerbatim}
}

\hrule height -\fboxrule\relax
\vspace{\nbsphinxcodecellspacing}

\makeatletter\setbox\nbsphinxpromptbox\box\voidb@x\makeatother

\begin{nbsphinxfancyoutput}

\noindent\sphinxincludegraphics[width=611\sphinxpxdimen,height=396\sphinxpxdimen]{{usage_get_started_6_0}.png}

\end{nbsphinxfancyoutput}

{

\kern-\sphinxverbatimsmallskipamount\kern-\baselineskip
\kern+\FrameHeightAdjust\kern-\fboxrule
\vspace{\nbsphinxcodecellspacing}
\sphinxsetup{VerbatimColor={named}{white}}

\sphinxsetup{VerbatimBorderColor={named}{nbsphinx-code-border}}
\begin{sphinxVerbatim}[commandchars=\\\{\}]
\llap{\color{nbsphinxout}[2]:\,\hspace{\fboxrule}\hspace{\fboxsep}}\PYGZlt{}ggplot: (8760784900369)\PYGZgt{}
\end{sphinxVerbatim}
}

To fit a minimal model to this data, call pc.MinimalHierarchicalModel:

{
\sphinxsetup{VerbatimColor={named}{nbsphinx-code-bg}}
\sphinxsetup{VerbatimBorderColor={named}{nbsphinx-code-border}}
\begin{sphinxVerbatim}[commandchars=\\\{\}]
\llap{\color{nbsphinxin}[3]:\,\hspace{\fboxrule}\hspace{\fboxsep}}\PYG{n}{hpc} \PYG{o}{=} \PYG{n}{pc}\PYG{o}{.}\PYG{n}{MinimalHierarchicalModel}\PYG{p}{(}\PYG{n}{art}\PYG{p}{,} \PYG{n}{weights}\PYG{o}{=}\PYG{l+s+s1}{\PYGZsq{}}\PYG{l+s+s1}{pca}\PYG{l+s+s1}{\PYGZsq{}}\PYG{p}{,} \PYG{n}{latent\PYGZus{}dimensions}\PYG{o}{=}\PYG{l+m+mi}{4}\PYG{p}{)}
\end{sphinxVerbatim}
}



%
{
\kern-\sphinxverbatimsmallskipamount\kern-\baselineskip
\kern+\FrameHeightAdjust\kern-\fboxrule
\vspace{\nbsphinxcodecellspacing}
\sphinxsetup{VerbatimBorderColor={named}{nbsphinx-code-border}}
\sphinxsetup{VerbatimColor={named}{nbsphinx-stderr}}
\fvset{hllines={, ,}}%
\begin{sphinxVerbatim}[commandchars=\\\{\}]
WARNING: Logging before flag parsing goes to stderr.
W0826 13:23:25.600336 140173786113856 estimator.py:1811] Using temporary folder as model directory: /tmp/tmpf6ungpzq
\end{sphinxVerbatim}
}
% The following \relax is needed to avoid problems with adjacent ANSI
% cells and some other stuff (e.g. bullet lists) following ANSI cells.
% See https://github.com/sphinx-doc/sphinx/issues/3594
\relax

{
\sphinxsetup{VerbatimColor={named}{nbsphinx-code-bg}}
\sphinxsetup{VerbatimBorderColor={named}{nbsphinx-code-border}}
\begin{sphinxVerbatim}[commandchars=\\\{\}]
\llap{\color{nbsphinxin}[4]:\,\hspace{\fboxrule}\hspace{\fboxsep}}\PYG{n}{hpc}\PYG{o}{.}\PYG{n}{train}\PYG{p}{(}\PYG{n}{max\PYGZus{}steps}\PYG{o}{=}\PYG{l+m+mf}{1e4}\PYG{p}{,} \PYG{n}{learning\PYGZus{}rate}\PYG{o}{=}\PYG{l+m+mi}{5}\PYG{p}{)}
\end{sphinxVerbatim}
}



%
{
\kern-\sphinxverbatimsmallskipamount\kern-\baselineskip
\kern+\FrameHeightAdjust\kern-\fboxrule
\vspace{\nbsphinxcodecellspacing}
\sphinxsetup{VerbatimBorderColor={named}{nbsphinx-code-border}}
\sphinxsetup{VerbatimColor={named}{nbsphinx-stderr}}
\fvset{hllines={, ,}}%
\begin{sphinxVerbatim}[commandchars=\\\{\}]
W0826 13:23:39.305405 140173786113856 estimator.py:1811] Using temporary folder as model directory: /tmp/tmpeo15sm9j
W0826 13:23:39.341255 140173786113856 deprecation.py:323] From /home/sflippl/.local/lib/python3.7/site-packages/tensorflow/python/training/training\_util.py:236: Variable.initialized\_value (from tensorflow.python.ops.variables) is deprecated and will be removed in a future version.
Instructions for updating:
Use Variable.read\_value. Variables in 2.X are initialized automatically both in eager and graph (inside tf.defun) contexts.
W0826 13:23:39.364500 140173786113856 deprecation\_wrapper.py:119] From /home/sflippl/.local/lib/python3.7/site-packages/predicode/hierarchical/interfaces/minimal\_model.py:59: The name tf.feature\_column.input\_layer is deprecated. Please use tf.compat.v1.feature\_column.input\_layer instead.

W0826 13:23:39.366652 140173786113856 deprecation.py:323] From /home/sflippl/.local/lib/python3.7/site-packages/tensorflow/python/feature\_column/feature\_column.py:205: NumericColumn.\_get\_dense\_tensor (from tensorflow.python.feature\_column.feature\_column\_v2) is deprecated and will be removed in a future version.
Instructions for updating:
The old \_FeatureColumn APIs are being deprecated. Please use the new FeatureColumn APIs instead.
W0826 13:23:39.368350 140173786113856 deprecation.py:323] From /home/sflippl/.local/lib/python3.7/site-packages/tensorflow/python/feature\_column/feature\_column.py:2115: NumericColumn.\_transform\_feature (from tensorflow.python.feature\_column.feature\_column\_v2) is deprecated and will be removed in a future version.
Instructions for updating:
The old \_FeatureColumn APIs are being deprecated. Please use the new FeatureColumn APIs instead.
W0826 13:23:39.380359 140173786113856 deprecation.py:323] From /home/sflippl/.local/lib/python3.7/site-packages/tensorflow/python/feature\_column/feature\_column.py:206: NumericColumn.\_variable\_shape (from tensorflow.python.feature\_column.feature\_column\_v2) is deprecated and will be removed in a future version.
Instructions for updating:
The old \_FeatureColumn APIs are being deprecated. Please use the new FeatureColumn APIs instead.
W0826 13:23:39.450145 140173786113856 deprecation.py:323] From /home/sflippl/.local/lib/python3.7/site-packages/predicode/hierarchical/interfaces/minimal\_model.py:63: dense (from tensorflow.python.layers.core) is deprecated and will be removed in a future version.
Instructions for updating:
Use keras.layers.dense instead.
W0826 13:23:39.457595 140173786113856 deprecation.py:506] From /home/sflippl/.local/lib/python3.7/site-packages/tensorflow/python/ops/init\_ops.py:1251: calling VarianceScaling.\_\_init\_\_ (from tensorflow.python.ops.init\_ops) with dtype is deprecated and will be removed in a future version.
Instructions for updating:
Call initializer instance with the dtype argument instead of passing it to the constructor
W0826 13:23:39.740041 140173786113856 deprecation\_wrapper.py:119] From /home/sflippl/.local/lib/python3.7/site-packages/predicode/hierarchical/interfaces/minimal\_model.py:35: The name tf.losses.mean\_squared\_error is deprecated. Please use tf.compat.v1.losses.mean\_squared\_error instead.

W0826 13:23:39.762232 140173786113856 deprecation.py:323] From /home/sflippl/.local/lib/python3.7/site-packages/tensorflow/python/ops/losses/losses\_impl.py:121: add\_dispatch\_support.<locals>.wrapper (from tensorflow.python.ops.array\_ops) is deprecated and will be removed in a future version.
Instructions for updating:
Use tf.where in 2.0, which has the same broadcast rule as np.where
W0826 13:23:39.789297 140173786113856 deprecation\_wrapper.py:119] From /home/sflippl/.local/lib/python3.7/site-packages/predicode/hierarchical/interfaces/minimal\_model.py:49: The name tf.train.GradientDescentOptimizer is deprecated. Please use tf.compat.v1.train.GradientDescentOptimizer instead.

W0826 13:23:39.790798 140173786113856 deprecation\_wrapper.py:119] From /home/sflippl/.local/lib/python3.7/site-packages/predicode/hierarchical/interfaces/minimal\_model.py:53: The name tf.train.get\_global\_step is deprecated. Please use tf.compat.v1.train.get\_global\_step instead.

W0826 13:23:40.310899 140173786113856 basic\_session\_run\_hooks.py:724] It seems that global step (tf.train.get\_global\_step) has not been increased. Current value (could be stable): 104 vs previous value: 104. You could increase the global step by passing tf.train.get\_global\_step() to Optimizer.apply\_gradients or Optimizer.minimize.
W0826 13:23:40.375190 140173786113856 basic\_session\_run\_hooks.py:724] It seems that global step (tf.train.get\_global\_step) has not been increased. Current value (could be stable): 201 vs previous value: 201. You could increase the global step by passing tf.train.get\_global\_step() to Optimizer.apply\_gradients or Optimizer.minimize.
W0826 13:23:40.425785 140173786113856 basic\_session\_run\_hooks.py:724] It seems that global step (tf.train.get\_global\_step) has not been increased. Current value (could be stable): 301 vs previous value: 301. You could increase the global step by passing tf.train.get\_global\_step() to Optimizer.apply\_gradients or Optimizer.minimize.
W0826 13:23:40.477852 140173786113856 basic\_session\_run\_hooks.py:724] It seems that global step (tf.train.get\_global\_step) has not been increased. Current value (could be stable): 401 vs previous value: 401. You could increase the global step by passing tf.train.get\_global\_step() to Optimizer.apply\_gradients or Optimizer.minimize.
W0826 13:23:40.614050 140173786113856 basic\_session\_run\_hooks.py:724] It seems that global step (tf.train.get\_global\_step) has not been increased. Current value (could be stable): 646 vs previous value: 646. You could increase the global step by passing tf.train.get\_global\_step() to Optimizer.apply\_gradients or Optimizer.minimize.
\end{sphinxVerbatim}
}
% The following \relax is needed to avoid problems with adjacent ANSI
% cells and some other stuff (e.g. bullet lists) following ANSI cells.
% See https://github.com/sphinx-doc/sphinx/issues/3594
\relax

{

\kern-\sphinxverbatimsmallskipamount\kern-\baselineskip
\kern+\FrameHeightAdjust\kern-\fboxrule
\vspace{\nbsphinxcodecellspacing}
\sphinxsetup{VerbatimColor={named}{white}}

\sphinxsetup{VerbatimBorderColor={named}{nbsphinx-code-border}}
\begin{sphinxVerbatim}[commandchars=\\\{\}]
\llap{\color{nbsphinxout}[4]:\,\hspace{\fboxrule}\hspace{\fboxsep}}\PYGZlt{}tensorflow\PYGZus{}estimator.python.estimator.estimator.Estimator at 0x7f7c700d2910\PYGZgt{}
\end{sphinxVerbatim}
}


\subsection{Outcome}
\label{\detokenize{usage/get_started:Outcome}}
{
\sphinxsetup{VerbatimColor={named}{nbsphinx-code-bg}}
\sphinxsetup{VerbatimBorderColor={named}{nbsphinx-code-border}}
\begin{sphinxVerbatim}[commandchars=\\\{\}]
\llap{\color{nbsphinxin}[5]:\,\hspace{\fboxrule}\hspace{\fboxsep}}\PYG{n}{hpc}\PYG{o}{.}\PYG{n}{evaluate}\PYG{p}{(}\PYG{p}{)}
\end{sphinxVerbatim}
}



%
{
\kern-\sphinxverbatimsmallskipamount\kern-\baselineskip
\kern+\FrameHeightAdjust\kern-\fboxrule
\vspace{\nbsphinxcodecellspacing}
\sphinxsetup{VerbatimBorderColor={named}{nbsphinx-code-border}}
\sphinxsetup{VerbatimColor={named}{nbsphinx-stderr}}
\fvset{hllines={, ,}}%
\begin{sphinxVerbatim}[commandchars=\\\{\}]
W0826 13:23:46.181127 140173786113856 deprecation.py:323] From /home/sflippl/.local/lib/python3.7/site-packages/tensorflow/python/training/saver.py:1276: checkpoint\_exists (from tensorflow.python.training.checkpoint\_management) is deprecated and will be removed in a future version.
Instructions for updating:
Use standard file APIs to check for files with this prefix.
\end{sphinxVerbatim}
}
% The following \relax is needed to avoid problems with adjacent ANSI
% cells and some other stuff (e.g. bullet lists) following ANSI cells.
% See https://github.com/sphinx-doc/sphinx/issues/3594
\relax

{

\kern-\sphinxverbatimsmallskipamount\kern-\baselineskip
\kern+\FrameHeightAdjust\kern-\fboxrule
\vspace{\nbsphinxcodecellspacing}
\sphinxsetup{VerbatimColor={named}{white}}

\sphinxsetup{VerbatimBorderColor={named}{nbsphinx-code-border}}
\begin{sphinxVerbatim}[commandchars=\\\{\}]
\llap{\color{nbsphinxout}[5]:\,\hspace{\fboxrule}\hspace{\fboxsep}}\PYGZob{}\PYGZsq{}loss\PYGZsq{}: 2.7815644e\PYGZhy{}05, \PYGZsq{}global\PYGZus{}step\PYGZsq{}: 10000\PYGZcb{}
\end{sphinxVerbatim}
}

{
\sphinxsetup{VerbatimColor={named}{nbsphinx-code-bg}}
\sphinxsetup{VerbatimBorderColor={named}{nbsphinx-code-border}}
\begin{sphinxVerbatim}[commandchars=\\\{\}]
\llap{\color{nbsphinxin}[6]:\,\hspace{\fboxrule}\hspace{\fboxsep}}\PYG{n}{lazytools}\PYG{o}{.}\PYG{n}{matrix\PYGZus{}heatmap}\PYG{p}{(}\PYG{n}{hpc}\PYG{o}{.}\PYG{n}{latent\PYGZus{}values}\PYG{p}{,} \PYG{n}{pole}\PYG{o}{=}\PYG{l+m+mi}{0}\PYG{p}{)}
\end{sphinxVerbatim}
}

\hrule height -\fboxrule\relax
\vspace{\nbsphinxcodecellspacing}

\makeatletter\setbox\nbsphinxpromptbox\box\voidb@x\makeatother

\begin{nbsphinxfancyoutput}

\noindent\sphinxincludegraphics[width=611\sphinxpxdimen,height=396\sphinxpxdimen]{{usage_get_started_12_0}.png}

\end{nbsphinxfancyoutput}

{

\kern-\sphinxverbatimsmallskipamount\kern-\baselineskip
\kern+\FrameHeightAdjust\kern-\fboxrule
\vspace{\nbsphinxcodecellspacing}
\sphinxsetup{VerbatimColor={named}{white}}

\sphinxsetup{VerbatimBorderColor={named}{nbsphinx-code-border}}
\begin{sphinxVerbatim}[commandchars=\\\{\}]
\llap{\color{nbsphinxout}[6]:\,\hspace{\fboxrule}\hspace{\fboxsep}}\PYGZlt{}ggplot: (8760775465813)\PYGZgt{}
\end{sphinxVerbatim}
}

{
\sphinxsetup{VerbatimColor={named}{nbsphinx-code-bg}}
\sphinxsetup{VerbatimBorderColor={named}{nbsphinx-code-border}}
\begin{sphinxVerbatim}[commandchars=\\\{\}]
\llap{\color{nbsphinxin}[7]:\,\hspace{\fboxrule}\hspace{\fboxsep}}\PYG{n}{lazytools}\PYG{o}{.}\PYG{n}{matrix\PYGZus{}heatmap}\PYG{p}{(}\PYG{n}{hpc}\PYG{o}{.}\PYG{n}{predict}\PYG{p}{(}\PYG{p}{)}\PYG{p}{,} \PYG{n}{pole}\PYG{o}{=}\PYG{l+m+mi}{0}\PYG{p}{)}
\end{sphinxVerbatim}
}



%
{
\kern-\sphinxverbatimsmallskipamount\kern-\baselineskip
\kern+\FrameHeightAdjust\kern-\fboxrule
\vspace{\nbsphinxcodecellspacing}
\sphinxsetup{VerbatimBorderColor={named}{nbsphinx-code-border}}
\sphinxsetup{VerbatimColor={named}{nbsphinx-stderr}}
\fvset{hllines={, ,}}%
\begin{sphinxVerbatim}[commandchars=\\\{\}]
W0826 13:23:53.079085 140173786113856 estimator.py:1000] Input graph does not use tf.data.Dataset or contain a QueueRunner. That means predict yields forever. This is probably a mistake.
\end{sphinxVerbatim}
}
% The following \relax is needed to avoid problems with adjacent ANSI
% cells and some other stuff (e.g. bullet lists) following ANSI cells.
% See https://github.com/sphinx-doc/sphinx/issues/3594
\relax

\hrule height -\fboxrule\relax
\vspace{\nbsphinxcodecellspacing}

\makeatletter\setbox\nbsphinxpromptbox\box\voidb@x\makeatother

\begin{nbsphinxfancyoutput}

\noindent\sphinxincludegraphics[width=611\sphinxpxdimen,height=396\sphinxpxdimen]{{usage_get_started_13_1}.png}

\end{nbsphinxfancyoutput}

{

\kern-\sphinxverbatimsmallskipamount\kern-\baselineskip
\kern+\FrameHeightAdjust\kern-\fboxrule
\vspace{\nbsphinxcodecellspacing}
\sphinxsetup{VerbatimColor={named}{white}}

\sphinxsetup{VerbatimBorderColor={named}{nbsphinx-code-border}}
\begin{sphinxVerbatim}[commandchars=\\\{\}]
\llap{\color{nbsphinxout}[7]:\,\hspace{\fboxrule}\hspace{\fboxsep}}\PYGZlt{}ggplot: (8760775572757)\PYGZgt{}
\end{sphinxVerbatim}
}


\chapter{Installation}
\label{\detokenize{usage/installation:Installation}}\label{\detokenize{usage/installation::doc}}
Install ‘predicode’ using pip and the command

\begin{sphinxVerbatim}[commandchars=\\\{\}]
pip install predicode
\end{sphinxVerbatim}

If you would like the newest development version, use pip to install the package from the \sphinxhref{github.com/sflippl/predicode}{GitHub} repository using the command

\begin{sphinxVerbatim}[commandchars=\\\{\}]
pip install git+https://github.com/sflippl/predicode
\end{sphinxVerbatim}


\chapter{Datasets}
\label{\detokenize{usage/datasets:Datasets}}\label{\detokenize{usage/datasets::doc}}
\sphinxhref{https://colab.research.google.com/github/sflippl/predicode/blob/master/usage/datasets.ipynb}{\sphinxincludegraphics{{/home/sflippl/Documents/predicode/predicode/docs/doctrees/images/b54294fd4278d22369403065806b2c772298d1fc/blue}.svg}}

{
\sphinxsetup{VerbatimColor={named}{nbsphinx-code-bg}}
\sphinxsetup{VerbatimBorderColor={named}{nbsphinx-code-border}}
\begin{sphinxVerbatim}[commandchars=\\\{\}]
\llap{\color{nbsphinxin}[1]:\,\hspace{\fboxrule}\hspace{\fboxsep}}\PYG{k}{try}\PYG{p}{:}
    \PYG{k+kn}{import} \PYG{n+nn}{predicode} \PYG{k}{as} \PYG{n+nn}{pc}
\PYG{k}{except}\PYG{p}{:}
    \PYG{o}{!}pip install git+https://github.com/sflippl/predicode
    \PYG{k+kn}{import} \PYG{n+nn}{predicode} \PYG{k}{as} \PYG{n+nn}{pc}
\PYG{k}{try}\PYG{p}{:}
    \PYG{k+kn}{import} \PYG{n+nn}{lazytools\PYGZus{}sflippl} \PYG{k}{as} \PYG{n+nn}{lazytools}
\PYG{k}{except}\PYG{p}{:}
    \PYG{o}{!}pip install git+https://github.com/sflippl/lazytools
    \PYG{k+kn}{import} \PYG{n+nn}{lazytools\PYGZus{}sflippl} \PYG{k}{as} \PYG{n+nn}{lazytools}
\end{sphinxVerbatim}
}


\section{Artificial Datasets}
\label{\detokenize{usage/datasets:Artificial-Datasets}}
Artificial datasets provide a simple example for how the algorithm works and an opportunity to study its analytical solutions.


\subsection{Decaying Multinormal Distribution}
\label{\detokenize{usage/datasets:Decaying-Multinormal-Distribution}}
The closed-form solution of a linear predictive coding model is given by a principal components analysis. A multinormal distribution allows for an easy model for such a solution. The class ‘DecayingMultiNormal’ models a high-dimensional input with decaying importance. Namely, the variance of the different principal components is specified using the decay constant ‘alpha’. Dimensionality of the input data is specified using ‘dimensions’ and sample size is specified by ‘samples’.

{
\sphinxsetup{VerbatimColor={named}{nbsphinx-code-bg}}
\sphinxsetup{VerbatimBorderColor={named}{nbsphinx-code-border}}
\begin{sphinxVerbatim}[commandchars=\\\{\}]
\llap{\color{nbsphinxin}[3]:\,\hspace{\fboxrule}\hspace{\fboxsep}}\PYG{n}{art\PYGZus{}data} \PYG{o}{=} \PYG{n}{pc}\PYG{o}{.}\PYG{n}{decaying\PYGZus{}multi\PYGZus{}normal}\PYG{p}{(}\PYG{n}{dimensions} \PYG{o}{=} \PYG{l+m+mi}{10}\PYG{p}{,}
                                    \PYG{n}{size} \PYG{o}{=} \PYG{l+m+mi}{10000}\PYG{p}{,}
                                    \PYG{n}{alpha} \PYG{o}{=} \PYG{l+m+mi}{1}\PYG{p}{)}
\PYG{k+kn}{import} \PYG{n+nn}{numpy} \PYG{k}{as} \PYG{n+nn}{np}
\end{sphinxVerbatim}
}

{
\sphinxsetup{VerbatimColor={named}{nbsphinx-code-bg}}
\sphinxsetup{VerbatimBorderColor={named}{nbsphinx-code-border}}
\begin{sphinxVerbatim}[commandchars=\\\{\}]
\llap{\color{nbsphinxin}[7]:\,\hspace{\fboxrule}\hspace{\fboxsep}}\PYG{n}{lazytools}\PYG{o}{.}\PYG{n}{matrix\PYGZus{}heatmap}\PYG{p}{(}\PYG{n}{art\PYGZus{}data}\PYG{p}{,} \PYG{n}{pole} \PYG{o}{=} \PYG{l+m+mi}{0}\PYG{p}{)}
\end{sphinxVerbatim}
}

\hrule height -\fboxrule\relax
\vspace{\nbsphinxcodecellspacing}

\makeatletter\setbox\nbsphinxpromptbox\box\voidb@x\makeatother

\begin{nbsphinxfancyoutput}

\noindent\sphinxincludegraphics[width=611\sphinxpxdimen,height=396\sphinxpxdimen]{{usage_datasets_8_0}.png}

\end{nbsphinxfancyoutput}

{

\kern-\sphinxverbatimsmallskipamount\kern-\baselineskip
\kern+\FrameHeightAdjust\kern-\fboxrule
\vspace{\nbsphinxcodecellspacing}
\sphinxsetup{VerbatimColor={named}{white}}

\sphinxsetup{VerbatimBorderColor={named}{nbsphinx-code-border}}
\begin{sphinxVerbatim}[commandchars=\\\{\}]
\llap{\color{nbsphinxout}[7]:\,\hspace{\fboxrule}\hspace{\fboxsep}}\PYGZlt{}ggplot: (8726908181521)\PYGZgt{}
\end{sphinxVerbatim}
}

{
\sphinxsetup{VerbatimColor={named}{nbsphinx-code-bg}}
\sphinxsetup{VerbatimBorderColor={named}{nbsphinx-code-border}}
\begin{sphinxVerbatim}[commandchars=\\\{\}]
\llap{\color{nbsphinxin}[8]:\,\hspace{\fboxrule}\hspace{\fboxsep}}\PYG{n}{lazytools}\PYG{o}{.}\PYG{n}{matrix\PYGZus{}heatmap}\PYG{p}{(}\PYG{n}{np}\PYG{o}{.}\PYG{n}{cov}\PYG{p}{(}\PYG{n}{art\PYGZus{}data}\PYG{o}{.}\PYG{n}{T}\PYG{p}{)}\PYG{p}{,} \PYG{n}{pole} \PYG{o}{=} \PYG{l+m+mi}{0}\PYG{p}{)}
\end{sphinxVerbatim}
}

\hrule height -\fboxrule\relax
\vspace{\nbsphinxcodecellspacing}

\makeatletter\setbox\nbsphinxpromptbox\box\voidb@x\makeatother

\begin{nbsphinxfancyoutput}

\noindent\sphinxincludegraphics[width=622\sphinxpxdimen,height=396\sphinxpxdimen]{{usage_datasets_9_0}.png}

\end{nbsphinxfancyoutput}

{

\kern-\sphinxverbatimsmallskipamount\kern-\baselineskip
\kern+\FrameHeightAdjust\kern-\fboxrule
\vspace{\nbsphinxcodecellspacing}
\sphinxsetup{VerbatimColor={named}{white}}

\sphinxsetup{VerbatimBorderColor={named}{nbsphinx-code-border}}
\begin{sphinxVerbatim}[commandchars=\\\{\}]
\llap{\color{nbsphinxout}[8]:\,\hspace{\fboxrule}\hspace{\fboxsep}}\PYGZlt{}ggplot: (8726884562885)\PYGZgt{}
\end{sphinxVerbatim}
}


\section{Image Datasets}
\label{\detokenize{usage/datasets:Image-Datasets}}
Image datasets are predominantly included as examples for the predictive coding algorithms under the ‘datasets’ module. Whereas their main purpose is being incorporated by the respective algorithms, ‘predicode’ allows for some functionality in exploring the datasets on their own. In particular, a number of images may be visualized using the pictures method (see Cifar-10 below).


\section{Cifar-10}
\label{\detokenize{usage/datasets:Cifar-10}}
Cifar-10 serves as a simple example dataset for basic predictive coding algorithms demonstrating static extraclassical effects.

For now, only the training dataset can be read in using the class \sphinxcode{\sphinxupquote{Cifar10}}

{
\sphinxsetup{VerbatimColor={named}{nbsphinx-code-bg}}
\sphinxsetup{VerbatimBorderColor={named}{nbsphinx-code-border}}
\begin{sphinxVerbatim}[commandchars=\\\{\}]
\llap{\color{nbsphinxin}[1]:\,\hspace{\fboxrule}\hspace{\fboxsep}}\PYG{k+kn}{import} \PYG{n+nn}{predicode} \PYG{k}{as} \PYG{n+nn}{pc}
\PYG{n}{cifar} \PYG{o}{=} \PYG{n}{pc}\PYG{o}{.}\PYG{n}{Cifar10}\PYG{p}{(}\PYG{p}{)}
\end{sphinxVerbatim}
}

This dataset may be explored by looking at the pictures along with their labels. This is possible in black-white and color.

{
\sphinxsetup{VerbatimColor={named}{nbsphinx-code-bg}}
\sphinxsetup{VerbatimBorderColor={named}{nbsphinx-code-border}}
\begin{sphinxVerbatim}[commandchars=\\\{\}]
\llap{\color{nbsphinxin}[2]:\,\hspace{\fboxrule}\hspace{\fboxsep}}\PYG{n}{cifar}\PYG{o}{.}\PYG{n}{pictures}\PYG{p}{(}\PYG{n}{subset} \PYG{o}{=} \PYG{n+nb}{range}\PYG{p}{(}\PYG{l+m+mi}{25}\PYG{p}{)}\PYG{p}{,} \PYG{n}{mode} \PYG{o}{=} \PYG{l+s+s1}{\PYGZsq{}}\PYG{l+s+s1}{bw}\PYG{l+s+s1}{\PYGZsq{}}\PYG{p}{)}
\end{sphinxVerbatim}
}

\hrule height -\fboxrule\relax
\vspace{\nbsphinxcodecellspacing}

\makeatletter\setbox\nbsphinxpromptbox\box\voidb@x\makeatother

\begin{nbsphinxfancyoutput}

\noindent\sphinxincludegraphics[width=523\sphinxpxdimen,height=669\sphinxpxdimen]{{usage_datasets_17_0}.png}

\end{nbsphinxfancyoutput}

{

\kern-\sphinxverbatimsmallskipamount\kern-\baselineskip
\kern+\FrameHeightAdjust\kern-\fboxrule
\vspace{\nbsphinxcodecellspacing}
\sphinxsetup{VerbatimColor={named}{white}}

\sphinxsetup{VerbatimBorderColor={named}{nbsphinx-code-border}}
\begin{sphinxVerbatim}[commandchars=\\\{\}]
\llap{\color{nbsphinxout}[2]:\,\hspace{\fboxrule}\hspace{\fboxsep}}\PYGZlt{}ggplot: (\PYGZhy{}9223363293114942036)\PYGZgt{}
\end{sphinxVerbatim}
}

{
\sphinxsetup{VerbatimColor={named}{nbsphinx-code-bg}}
\sphinxsetup{VerbatimBorderColor={named}{nbsphinx-code-border}}
\begin{sphinxVerbatim}[commandchars=\\\{\}]
\llap{\color{nbsphinxin}[3]:\,\hspace{\fboxrule}\hspace{\fboxsep}}\PYG{n}{cifar}\PYG{o}{.}\PYG{n}{pictures}\PYG{p}{(}\PYG{n}{subset} \PYG{o}{=} \PYG{n+nb}{range}\PYG{p}{(}\PYG{l+m+mi}{25}\PYG{p}{)}\PYG{p}{,} \PYG{n}{mode} \PYG{o}{=} \PYG{l+s+s1}{\PYGZsq{}}\PYG{l+s+s1}{color}\PYG{l+s+s1}{\PYGZsq{}}\PYG{p}{)}
\end{sphinxVerbatim}
}

\hrule height -\fboxrule\relax
\vspace{\nbsphinxcodecellspacing}

\makeatletter\setbox\nbsphinxpromptbox\box\voidb@x\makeatother

\begin{nbsphinxfancyoutput}

\noindent\sphinxincludegraphics[width=523\sphinxpxdimen,height=669\sphinxpxdimen]{{usage_datasets_18_0}.png}

\end{nbsphinxfancyoutput}

{

\kern-\sphinxverbatimsmallskipamount\kern-\baselineskip
\kern+\FrameHeightAdjust\kern-\fboxrule
\vspace{\nbsphinxcodecellspacing}
\sphinxsetup{VerbatimColor={named}{white}}

\sphinxsetup{VerbatimBorderColor={named}{nbsphinx-code-border}}
\begin{sphinxVerbatim}[commandchars=\\\{\}]
\llap{\color{nbsphinxout}[3]:\,\hspace{\fboxrule}\hspace{\fboxsep}}\PYGZlt{}ggplot: (8743719730980)\PYGZgt{}
\end{sphinxVerbatim}
}

This builds upon the underlying data frame that contains the RGB values for the color and the black-white pictures:

{
\sphinxsetup{VerbatimColor={named}{nbsphinx-code-bg}}
\sphinxsetup{VerbatimBorderColor={named}{nbsphinx-code-border}}
\begin{sphinxVerbatim}[commandchars=\\\{\}]
\llap{\color{nbsphinxin}[4]:\,\hspace{\fboxrule}\hspace{\fboxsep}}\PYG{n}{cifar}\PYG{o}{.}\PYG{n}{rgb\PYGZus{}dataframe}\PYG{p}{(}\PYG{n}{subset} \PYG{o}{=} \PYG{n+nb}{range}\PYG{p}{(}\PYG{l+m+mi}{1}\PYG{p}{)}\PYG{p}{)}\PYG{o}{.}\PYG{n}{head}\PYG{p}{(}\PYG{p}{)}
\end{sphinxVerbatim}
}

{

\kern-\sphinxverbatimsmallskipamount\kern-\baselineskip
\kern+\FrameHeightAdjust\kern-\fboxrule
\vspace{\nbsphinxcodecellspacing}
\sphinxsetup{VerbatimColor={named}{white}}

\sphinxsetup{VerbatimBorderColor={named}{nbsphinx-code-border}}
\begin{sphinxVerbatim}[commandchars=\\\{\}]
\llap{\color{nbsphinxout}[4]:\,\hspace{\fboxrule}\hspace{\fboxsep}}   image\PYGZus{}id  x  y   r   g   b         bw      rgb   rgb\PYGZus{}bw
0         0  0  0  59  62  63  61.333333  \PYGZsh{}3b3e3f  \PYGZsh{}3d3d3d
1         0  1  0  43  46  45  44.666667  \PYGZsh{}2b2e2d  \PYGZsh{}2c2c2c
2         0  2  0  50  48  43  47.000000  \PYGZsh{}32302b  \PYGZsh{}2f2f2f
3         0  3  0  68  54  42  54.666667  \PYGZsh{}44362a  \PYGZsh{}363636
4         0  4  0  98  73  52  74.333333  \PYGZsh{}624934  \PYGZsh{}4a4a4a
\end{sphinxVerbatim}
}


\chapter{Minimal Model}
\label{\detokenize{usage/minimal_model:Minimal-Model}}\label{\detokenize{usage/minimal_model::doc}}
\sphinxhref{https://colab.research.google.com/github/sflippl/predicode/blob/master/usage/minimal\_model.ipynb}{\sphinxincludegraphics{{/home/sflippl/Documents/predicode/predicode/docs/doctrees/images/1cbef8f858ac0e12d0cf2cb67076bcaff7e5a3c8/blue}.svg}}

{
\sphinxsetup{VerbatimColor={named}{nbsphinx-code-bg}}
\sphinxsetup{VerbatimBorderColor={named}{nbsphinx-code-border}}
\begin{sphinxVerbatim}[commandchars=\\\{\}]
\llap{\color{nbsphinxin}[1]:\,\hspace{\fboxrule}\hspace{\fboxsep}}\PYG{k}{try}\PYG{p}{:}
    \PYG{k+kn}{import} \PYG{n+nn}{predicode} \PYG{k}{as} \PYG{n+nn}{pc}
\PYG{k}{except}\PYG{p}{:}
    \PYG{o}{!}pip install git+https://github.com/sflippl/predicode
    \PYG{k+kn}{import} \PYG{n+nn}{predicode} \PYG{k}{as} \PYG{n+nn}{pc}
\PYG{k}{try}\PYG{p}{:}
    \PYG{k+kn}{import} \PYG{n+nn}{lazytools\PYGZus{}sflippl} \PYG{k}{as} \PYG{n+nn}{lazytools}
\PYG{k}{except}\PYG{p}{:}
    \PYG{o}{!}pip install git+https://github.com/sflippl/lazytools
    \PYG{k+kn}{import} \PYG{n+nn}{lazytools\PYGZus{}sflippl} \PYG{k}{as} \PYG{n+nn}{lazytools}
\PYG{k+kn}{import} \PYG{n+nn}{pandas} \PYG{k}{as} \PYG{n+nn}{pd}
\PYG{k+kn}{import} \PYG{n+nn}{numpy} \PYG{k}{as} \PYG{n+nn}{np}
\PYG{k+kn}{import} \PYG{n+nn}{plotnine} \PYG{k}{as} \PYG{n+nn}{gg}
\PYG{k+kn}{import} \PYG{n+nn}{tensorflow} \PYG{k}{as} \PYG{n+nn}{tf}
\PYG{n}{tf}\PYG{o}{.}\PYG{n}{logging}\PYG{o}{.}\PYG{n}{set\PYGZus{}verbosity}\PYG{p}{(}\PYG{n}{tf}\PYG{o}{.}\PYG{n}{logging}\PYG{o}{.}\PYG{n}{ERROR}\PYG{p}{)}
\PYG{n}{tf}\PYG{o}{.}\PYG{n}{random}\PYG{o}{.}\PYG{n}{set\PYGZus{}random\PYGZus{}seed}\PYG{p}{(}\PYG{l+m+mi}{2885}\PYG{p}{)}
\PYG{n}{np}\PYG{o}{.}\PYG{n}{random}\PYG{o}{.}\PYG{n}{seed}\PYG{p}{(}\PYG{l+m+mi}{2885}\PYG{p}{)}
\end{sphinxVerbatim}
}

‘predicode’ contains several %
\begin{footnote}[1]\sphinxAtStartFootnote
so far one
%
\end{footnote} high-level interfaces to the more general hierarchical model %
\begin{footnote}[2]\sphinxAtStartFootnote
not yet a thing
%
\end{footnote}. We will use the minimal model, consisting of an input layer and one latent layer with a specified number of dimensions as an example. As an example, we will use an artificial dataset as presented in the previous chapter.

{
\sphinxsetup{VerbatimColor={named}{nbsphinx-code-bg}}
\sphinxsetup{VerbatimBorderColor={named}{nbsphinx-code-border}}
\begin{sphinxVerbatim}[commandchars=\\\{\}]
\llap{\color{nbsphinxin}[2]:\,\hspace{\fboxrule}\hspace{\fboxsep}}\PYG{n}{art} \PYG{o}{=} \PYG{n}{pc}\PYG{o}{.}\PYG{n}{decaying\PYGZus{}multi\PYGZus{}normal}\PYG{p}{(}\PYG{n}{dimensions} \PYG{o}{=} \PYG{l+m+mi}{10}\PYG{p}{,}
                               \PYG{n}{size} \PYG{o}{=} \PYG{l+m+mi}{100}\PYG{p}{)}
\end{sphinxVerbatim}
}

The minimal model can be fitted by providing input data and the number of latent dimensions to the class ‘pc.MinimalHierarchicalModel’:

{
\sphinxsetup{VerbatimColor={named}{nbsphinx-code-bg}}
\sphinxsetup{VerbatimBorderColor={named}{nbsphinx-code-border}}
\begin{sphinxVerbatim}[commandchars=\\\{\}]
\llap{\color{nbsphinxin}[3]:\,\hspace{\fboxrule}\hspace{\fboxsep}}\PYG{n}{hpc} \PYG{o}{=} \PYG{n}{pc}\PYG{o}{.}\PYG{n}{MinimalHierarchicalModel}\PYG{p}{(}\PYG{n}{input\PYGZus{}data} \PYG{o}{=} \PYG{n}{art}\PYG{p}{,}
                                  \PYG{n}{latent\PYGZus{}dimensions} \PYG{o}{=} \PYG{l+m+mi}{4}\PYG{p}{)}
\end{sphinxVerbatim}
}


\section{State Estimation}
\label{\detokenize{usage/minimal_model:State-Estimation}}
By default the weights of the minimal model are initialized as the first PCA components. This is the optimal solution for the minimal model and can be used to study state estimation. Since a predictive coding model can improve either by adapting its states or its weights, we first need to specify which of the two can currently be modified:

{
\sphinxsetup{VerbatimColor={named}{nbsphinx-code-bg}}
\sphinxsetup{VerbatimBorderColor={named}{nbsphinx-code-border}}
\begin{sphinxVerbatim}[commandchars=\\\{\}]
\llap{\color{nbsphinxin}[4]:\,\hspace{\fboxrule}\hspace{\fboxsep}}\PYG{n}{hpc}\PYG{o}{.}\PYG{n}{activate}\PYG{p}{(}\PYG{l+s+s1}{\PYGZsq{}}\PYG{l+s+s1}{state}\PYG{l+s+s1}{\PYGZsq{}}\PYG{p}{)}
\end{sphinxVerbatim}
}

(Technically, this would not have been necessary.) States are activated by default.

The minimal model can then be trained in order to extract the latent values.

{
\sphinxsetup{VerbatimColor={named}{nbsphinx-code-bg}}
\sphinxsetup{VerbatimBorderColor={named}{nbsphinx-code-border}}
\begin{sphinxVerbatim}[commandchars=\\\{\}]
\llap{\color{nbsphinxin}[5]:\,\hspace{\fboxrule}\hspace{\fboxsep}}\PYG{n}{hpc}\PYG{o}{.}\PYG{n}{train}\PYG{p}{(}\PYG{n}{max\PYGZus{}steps}\PYG{o}{=}\PYG{l+m+mf}{1e3}\PYG{p}{)}
\end{sphinxVerbatim}
}

{

\kern-\sphinxverbatimsmallskipamount\kern-\baselineskip
\kern+\FrameHeightAdjust\kern-\fboxrule
\vspace{\nbsphinxcodecellspacing}
\sphinxsetup{VerbatimColor={named}{white}}

\sphinxsetup{VerbatimBorderColor={named}{nbsphinx-code-border}}
\begin{sphinxVerbatim}[commandchars=\\\{\}]
\llap{\color{nbsphinxout}[5]:\,\hspace{\fboxrule}\hspace{\fboxsep}}\PYGZlt{}tensorflow\PYGZus{}estimator.python.estimator.estimator.Estimator at 0x7f9c101a6fd0\PYGZgt{}
\end{sphinxVerbatim}
}

The usual Tensorflow framework can be used in this context:

{
\sphinxsetup{VerbatimColor={named}{nbsphinx-code-bg}}
\sphinxsetup{VerbatimBorderColor={named}{nbsphinx-code-border}}
\begin{sphinxVerbatim}[commandchars=\\\{\}]
\llap{\color{nbsphinxin}[6]:\,\hspace{\fboxrule}\hspace{\fboxsep}}\PYG{n}{hpc}\PYG{o}{.}\PYG{n}{evaluate}\PYG{p}{(}\PYG{p}{)}
\end{sphinxVerbatim}
}

{

\kern-\sphinxverbatimsmallskipamount\kern-\baselineskip
\kern+\FrameHeightAdjust\kern-\fboxrule
\vspace{\nbsphinxcodecellspacing}
\sphinxsetup{VerbatimColor={named}{white}}

\sphinxsetup{VerbatimBorderColor={named}{nbsphinx-code-border}}
\begin{sphinxVerbatim}[commandchars=\\\{\}]
\llap{\color{nbsphinxout}[6]:\,\hspace{\fboxrule}\hspace{\fboxsep}}\PYGZob{}\PYGZsq{}loss\PYGZsq{}: 0.0026129463, \PYGZsq{}global\PYGZus{}step\PYGZsq{}: 1000\PYGZcb{}
\end{sphinxVerbatim}
}

The method ‘latent\_values’ allows for the extraction of the latent values, which we here visualize using a heatmap:

{
\sphinxsetup{VerbatimColor={named}{nbsphinx-code-bg}}
\sphinxsetup{VerbatimBorderColor={named}{nbsphinx-code-border}}
\begin{sphinxVerbatim}[commandchars=\\\{\}]
\llap{\color{nbsphinxin}[7]:\,\hspace{\fboxrule}\hspace{\fboxsep}}\PYG{n}{latent\PYGZus{}values} \PYG{o}{=} \PYG{n}{hpc}\PYG{o}{.}\PYG{n}{latent\PYGZus{}values}
\PYG{n}{lazytools}\PYG{o}{.}\PYG{n}{matrix\PYGZus{}heatmap}\PYG{p}{(}\PYG{n}{latent\PYGZus{}values}\PYG{p}{,} \PYG{n}{pole} \PYG{o}{=} \PYG{l+m+mi}{0}\PYG{p}{)}
\end{sphinxVerbatim}
}

\hrule height -\fboxrule\relax
\vspace{\nbsphinxcodecellspacing}

\makeatletter\setbox\nbsphinxpromptbox\box\voidb@x\makeatother

\begin{nbsphinxfancyoutput}

\noindent\sphinxincludegraphics[width=611\sphinxpxdimen,height=396\sphinxpxdimen]{{usage_minimal_model_16_0}.png}

\end{nbsphinxfancyoutput}

{

\kern-\sphinxverbatimsmallskipamount\kern-\baselineskip
\kern+\FrameHeightAdjust\kern-\fboxrule
\vspace{\nbsphinxcodecellspacing}
\sphinxsetup{VerbatimColor={named}{white}}

\sphinxsetup{VerbatimBorderColor={named}{nbsphinx-code-border}}
\begin{sphinxVerbatim}[commandchars=\\\{\}]
\llap{\color{nbsphinxout}[7]:\,\hspace{\fboxrule}\hspace{\fboxsep}}\PYGZlt{}ggplot: (8769264820429)\PYGZgt{}
\end{sphinxVerbatim}
}

Similarly the predictions generated by the latent values can be studied:

{
\sphinxsetup{VerbatimColor={named}{nbsphinx-code-bg}}
\sphinxsetup{VerbatimBorderColor={named}{nbsphinx-code-border}}
\begin{sphinxVerbatim}[commandchars=\\\{\}]
\llap{\color{nbsphinxin}[8]:\,\hspace{\fboxrule}\hspace{\fboxsep}}\PYG{n}{predictions} \PYG{o}{=} \PYG{n}{hpc}\PYG{o}{.}\PYG{n}{predict}\PYG{p}{(}\PYG{p}{)}
\PYG{n}{lazytools}\PYG{o}{.}\PYG{n}{matrix\PYGZus{}heatmap}\PYG{p}{(}\PYG{n}{predictions}\PYG{p}{,} \PYG{n}{pole} \PYG{o}{=} \PYG{l+m+mi}{0}\PYG{p}{)}
\end{sphinxVerbatim}
}

\hrule height -\fboxrule\relax
\vspace{\nbsphinxcodecellspacing}

\makeatletter\setbox\nbsphinxpromptbox\box\voidb@x\makeatother

\begin{nbsphinxfancyoutput}

\noindent\sphinxincludegraphics[width=611\sphinxpxdimen,height=396\sphinxpxdimen]{{usage_minimal_model_18_0}.png}

\end{nbsphinxfancyoutput}

{

\kern-\sphinxverbatimsmallskipamount\kern-\baselineskip
\kern+\FrameHeightAdjust\kern-\fboxrule
\vspace{\nbsphinxcodecellspacing}
\sphinxsetup{VerbatimColor={named}{white}}

\sphinxsetup{VerbatimBorderColor={named}{nbsphinx-code-border}}
\begin{sphinxVerbatim}[commandchars=\\\{\}]
\llap{\color{nbsphinxout}[8]:\,\hspace{\fboxrule}\hspace{\fboxsep}}\PYGZlt{}ggplot: (8769230529841)\PYGZgt{}
\end{sphinxVerbatim}
}


\section{State Learning Curve}
\label{\detokenize{usage/minimal_model:State-Learning-Curve}}
Learning curves may also be simulated. Here, we take a single sample from the previous model and simulate its convergence.

{
\sphinxsetup{VerbatimColor={named}{nbsphinx-code-bg}}
\sphinxsetup{VerbatimBorderColor={named}{nbsphinx-code-border}}
\begin{sphinxVerbatim}[commandchars=\\\{\}]
\llap{\color{nbsphinxin}[9]:\,\hspace{\fboxrule}\hspace{\fboxsep}}\PYG{n}{curve\PYGZus{}hpc} \PYG{o}{=} \PYG{n}{pc}\PYG{o}{.}\PYG{n}{MinimalHierarchicalModel}\PYG{p}{(}\PYG{n}{input\PYGZus{}data}\PYG{o}{=}\PYG{n}{art}\PYG{p}{[}\PYG{n+nb}{range}\PYG{p}{(}\PYG{l+m+mi}{1}\PYG{p}{)}\PYG{p}{,} \PYG{p}{:}\PYG{p}{]}\PYG{p}{,}
                                        \PYG{n}{weights}\PYG{o}{=}\PYG{n}{hpc}\PYG{o}{.}\PYG{n}{weights}\PYG{p}{,}
                                        \PYG{n}{latent\PYGZus{}dimensions}\PYG{o}{=}\PYG{l+m+mi}{4}\PYG{p}{)}
\end{sphinxVerbatim}
}

{
\sphinxsetup{VerbatimColor={named}{nbsphinx-code-bg}}
\sphinxsetup{VerbatimBorderColor={named}{nbsphinx-code-border}}
\begin{sphinxVerbatim}[commandchars=\\\{\}]
\llap{\color{nbsphinxin}[10]:\,\hspace{\fboxrule}\hspace{\fboxsep}}\PYG{n}{learning\PYGZus{}curve} \PYG{o}{=} \PYG{n}{curve\PYGZus{}hpc}\PYG{o}{.}\PYG{n}{learning\PYGZus{}curve}\PYG{p}{(}\PYG{n}{steps}\PYG{o}{=}\PYG{l+m+mi}{100}\PYG{p}{,} \PYG{n}{resolution}\PYG{o}{=}\PYG{l+m+mi}{1}\PYG{p}{)}
\end{sphinxVerbatim}
}



%
{
\kern-\sphinxverbatimsmallskipamount\kern-\baselineskip
\kern+\FrameHeightAdjust\kern-\fboxrule
\vspace{\nbsphinxcodecellspacing}
\sphinxsetup{VerbatimBorderColor={named}{nbsphinx-code-border}}
\sphinxsetup{VerbatimColor={named}{nbsphinx-stderr}}
\fvset{hllines={, ,}}%
\begin{sphinxVerbatim}[commandchars=\\\{\}]
100\%|██████████| 100/100 [00:44<00:00,  2.11it/s]
\end{sphinxVerbatim}
}
% The following \relax is needed to avoid problems with adjacent ANSI
% cells and some other stuff (e.g. bullet lists) following ANSI cells.
% See https://github.com/sphinx-doc/sphinx/issues/3594
\relax

(Note that this is at the time extremely slow.)

{
\sphinxsetup{VerbatimColor={named}{nbsphinx-code-bg}}
\sphinxsetup{VerbatimBorderColor={named}{nbsphinx-code-border}}
\begin{sphinxVerbatim}[commandchars=\\\{\}]
\llap{\color{nbsphinxin}[11]:\,\hspace{\fboxrule}\hspace{\fboxsep}}\PYG{n}{df\PYGZus{}learning\PYGZus{}curve} \PYG{o}{=} \PYG{n}{lazytools}\PYG{o}{.}\PYG{n}{array\PYGZus{}to\PYGZus{}dataframe}\PYG{p}{(}\PYG{n}{learning\PYGZus{}curve}\PYG{p}{)}
\end{sphinxVerbatim}
}

{
\sphinxsetup{VerbatimColor={named}{nbsphinx-code-bg}}
\sphinxsetup{VerbatimBorderColor={named}{nbsphinx-code-border}}
\begin{sphinxVerbatim}[commandchars=\\\{\}]
\llap{\color{nbsphinxin}[12]:\,\hspace{\fboxrule}\hspace{\fboxsep}}\PYG{n}{df\PYGZus{}learning\PYGZus{}curve}\PYG{p}{[}\PYG{l+s+s1}{\PYGZsq{}}\PYG{l+s+s1}{dim1}\PYG{l+s+s1}{\PYGZsq{}}\PYG{p}{]} \PYG{o}{=} \PYG{p}{[}\PYG{n+nb}{str}\PYG{p}{(}\PYG{n}{dim1}\PYG{p}{)} \PYG{k}{for} \PYG{n}{dim1} \PYG{o+ow}{in} \PYG{n}{df\PYGZus{}learning\PYGZus{}curve}\PYG{p}{[}\PYG{l+s+s1}{\PYGZsq{}}\PYG{l+s+s1}{dim1}\PYG{l+s+s1}{\PYGZsq{}}\PYG{p}{]}\PYG{p}{]}
\PYG{p}{(}\PYG{n}{gg}\PYG{o}{.}\PYG{n}{ggplot}\PYG{p}{(}\PYG{n}{df\PYGZus{}learning\PYGZus{}curve}\PYG{p}{,} \PYG{n}{gg}\PYG{o}{.}\PYG{n}{aes}\PYG{p}{(}\PYG{n}{x}\PYG{o}{=}\PYG{l+s+s1}{\PYGZsq{}}\PYG{l+s+s1}{dim2}\PYG{l+s+s1}{\PYGZsq{}}\PYG{p}{,} \PYG{n}{y}\PYG{o}{=}\PYG{l+s+s1}{\PYGZsq{}}\PYG{l+s+s1}{array}\PYG{l+s+s1}{\PYGZsq{}}\PYG{p}{,} \PYG{n}{color}\PYG{o}{=}\PYG{l+s+s1}{\PYGZsq{}}\PYG{l+s+s1}{dim1}\PYG{l+s+s1}{\PYGZsq{}}\PYG{p}{,} \PYG{n}{group}\PYG{o}{=}\PYG{l+s+s1}{\PYGZsq{}}\PYG{l+s+s1}{dim1}\PYG{l+s+s1}{\PYGZsq{}}\PYG{p}{)}\PYG{p}{)} \PYG{o}{+}
    \PYG{n}{gg}\PYG{o}{.}\PYG{n}{geom\PYGZus{}line}\PYG{p}{(}\PYG{p}{)} \PYG{o}{+}
    \PYG{n}{gg}\PYG{o}{.}\PYG{n}{xlab}\PYG{p}{(}\PYG{l+s+s1}{\PYGZsq{}}\PYG{l+s+s1}{Step}\PYG{l+s+s1}{\PYGZsq{}}\PYG{p}{)} \PYG{o}{+}
    \PYG{n}{gg}\PYG{o}{.}\PYG{n}{ylab}\PYG{p}{(}\PYG{l+s+s1}{\PYGZsq{}}\PYG{l+s+s1}{Estimated Value}\PYG{l+s+s1}{\PYGZsq{}}\PYG{p}{)} \PYG{o}{+}
    \PYG{n}{gg}\PYG{o}{.}\PYG{n}{labs}\PYG{p}{(}\PYG{n}{color}\PYG{o}{=}\PYG{l+s+s1}{\PYGZsq{}}\PYG{l+s+s1}{Latent Dimension}\PYG{l+s+s1}{\PYGZsq{}}\PYG{p}{)} \PYG{o}{+}
    \PYG{n}{gg}\PYG{o}{.}\PYG{n}{theme\PYGZus{}bw}\PYG{p}{(}\PYG{p}{)}\PYG{p}{)}
\end{sphinxVerbatim}
}

\hrule height -\fboxrule\relax
\vspace{\nbsphinxcodecellspacing}

\makeatletter\setbox\nbsphinxpromptbox\box\voidb@x\makeatother

\begin{nbsphinxfancyoutput}

\noindent\sphinxincludegraphics[width=752\sphinxpxdimen,height=433\sphinxpxdimen]{{usage_minimal_model_25_0}.png}

\end{nbsphinxfancyoutput}

{

\kern-\sphinxverbatimsmallskipamount\kern-\baselineskip
\kern+\FrameHeightAdjust\kern-\fboxrule
\vspace{\nbsphinxcodecellspacing}
\sphinxsetup{VerbatimColor={named}{white}}

\sphinxsetup{VerbatimBorderColor={named}{nbsphinx-code-border}}
\begin{sphinxVerbatim}[commandchars=\\\{\}]
\llap{\color{nbsphinxout}[12]:\,\hspace{\fboxrule}\hspace{\fboxsep}}\PYGZlt{}ggplot: (8769230573945)\PYGZgt{}
\end{sphinxVerbatim}
}

\DUrole{xref,std,std-ref}{modindex}



\renewcommand{\indexname}{Index}
\printindex
\end{document}